\documentclass[aspectratio=169]{beamer}
\usetheme{Madrid}
\usecolortheme{whale}

\usepackage[utf8]{inputenc}
\usepackage[T1]{fontenc}
\usepackage[french]{babel}
\usepackage{graphicx}
\usepackage{tikz}
\usepackage{listings}
\usepackage{booktabs}

% Custom colors
\definecolor{polyblue}{RGB}{0, 82, 147}
\definecolor{darkblue}{RGB}{0, 51, 102}
\setbeamercolor{structure}{fg=polyblue}
\setbeamercolor{title}{fg=white,bg=polyblue}
\setbeamercolor{frametitle}{fg=white,bg=polyblue}

% Remove navigation symbols
\setbeamertemplate{navigation symbols}{}

% Title info
\title[Vultester]{Vultester}
\subtitle{Système Expert de Détection de Vulnérabilités Serveur}
\author{Mohamed Aziz Mansour}
\institute{École Polytechnique de Sousse\\Tuteur: F. SBIAA}
\date{4INFO 2025-2026}

\begin{document}

% ===== SLIDE 1: Title =====
\begin{frame}
\titlepage
\end{frame}

% ===== SLIDE 2: Plan =====
\begin{frame}{Plan}
\tableofcontents
\end{frame}

% ===== SECTION 1 =====
\section{Introduction}

\begin{frame}{Contexte}
\begin{columns}
\column{0.6\textwidth}
\begin{itemize}
    \item \textbf{Solution}: Système expert pour détecter automatiquement les failles de sécurité
    \item \textbf{Objectif}: Analyser la configuration et recommander des correctifs
\end{itemize}

\column{0.4\textwidth}
\begin{tikzpicture}[scale=0.8]
    \node[draw, fill=red!20, rounded corners, minimum width=2.5cm, minimum height=1cm] at (0,2) {Menaces};
    \node[draw, fill=blue!20, rounded corners, minimum width=2.5cm, minimum height=1cm] at (0,0) {Serveur};
    \node[draw, fill=green!20, rounded corners, minimum width=2.5cm, minimum height=1cm] at (0,-2) {Vultester};
    \draw[->, thick] (0,1.4) -- (0,0.6);
    \draw[->, thick, dashed] (0,-0.6) -- (0,-1.4);
\end{tikzpicture}
\end{columns}
\end{frame}

% ===== SECTION 2 =====
\section{Architecture du Système}

\begin{frame}{Architecture Globale}
\begin{center}
\begin{tikzpicture}[node distance=2cm, scale=0.9, transform shape]
    % Frontend
    \node[draw, fill=blue!20, rounded corners, minimum width=3cm, minimum height=1.5cm] (frontend) at (0,0) {\textbf{Frontend}\\React + Tailwind};
    
    % Backend
    \node[draw, fill=green!20, rounded corners, minimum width=3cm, minimum height=1.5cm] (backend) at (5,0) {\textbf{Backend}\\Flask API};
    
    % Expert System
    \node[draw, fill=orange!20, rounded corners, minimum width=3cm, minimum height=1.5cm] (expert) at (10,0) {\textbf{Moteur}\\d'Inférence};
    
    % Knowledge Base
    \node[draw, fill=purple!20, rounded corners, minimum width=3cm, minimum height=1.5cm] (kb) at (10,-3) {\textbf{Base de}\\Connaissances};
    
    % Arrows
    \draw[->, thick] (frontend) -- node[above] {HTTP} (backend);
    \draw[->, thick] (backend) -- (expert);
    \draw[<->, thick] (expert) -- (kb);
\end{tikzpicture}
\end{center}
\end{frame}

\begin{frame}{Base de Connaissances}
\begin{columns}
\column{0.5\textwidth}
\textbf{50 Règles Expertes}
\begin{itemize}
    \item PORT (10 règles) - Ports ouverts
    \item SSL (8 règles) - Certificats
    \item SSH (8 règles) - Configuration SSH
    \item PERM (8 règles) - Permissions
    \item SOFT (8 règles) - Logiciels
    \item NET (8 règles) - Réseau
\end{itemize}

\column{0.5\textwidth}
\textbf{4 Niveaux de Sévérité}
\begin{itemize}
    \item[\textcolor{red}{$\bullet$}] \textbf{CRITICAL} - Action immédiate
    \item[\textcolor{orange}{$\bullet$}] \textbf{DANGEROUS} - Risque élevé
    \item[\textcolor{yellow}{$\bullet$}] \textbf{WARNING} - Attention
    \item[\textcolor{blue}{$\bullet$}] \textbf{INFO} - Information
\end{itemize}
\end{columns}
\end{frame}

% ===== SECTION 3 =====
\section{Moteurs d'Inférence}

\begin{frame}{Chaînage Avant (Forward Chaining)}
\begin{columns}
\column{0.5\textwidth}
\textbf{Principe:}
\begin{enumerate}
    \item Partir des faits initiaux
    \item Chercher les règles applicables
    \item Déduire de nouveaux faits
    \item Répéter jusqu'à saturation
\end{enumerate}

\column{0.5\textwidth}
\begin{tikzpicture}[scale=0.7]
    \node[draw, circle, fill=blue!20] (f1) at (0,0) {F1};
    \node[draw, circle, fill=blue!20] (f2) at (2,0) {F2};
    \node[draw, rectangle, fill=yellow!20] (r1) at (1,-1.5) {R1};
    \node[draw, circle, fill=green!20] (f3) at (1,-3) {F3};
    
    \draw[->, thick] (f1) -- (r1);
    \draw[->, thick] (f2) -- (r1);
    \draw[->, thick] (r1) -- (f3);
\end{tikzpicture}
\end{columns}
\end{frame}

\begin{frame}{Chaînage Arrière (Backward Chaining)}
\begin{columns}
\column{0.5\textwidth}
\textbf{Principe:}
\begin{enumerate}
    \item Partir d'un but à prouver
    \item Chercher les règles qui concluent ce but
    \item Vérifier les conditions (récursivement)
    \item Valider ou invalider le but
\end{enumerate}

\column{0.5\textwidth}
\begin{tikzpicture}[scale=0.7]
    \node[draw, circle, fill=green!20] (goal) at (1,0) {But};
    \node[draw, rectangle, fill=yellow!20] (r1) at (1,-1.5) {R1};
    \node[draw, circle, fill=blue!20] (c1) at (0,-3) {C1};
    \node[draw, circle, fill=blue!20] (c2) at (2,-3) {C2};
    
    \draw[->, thick, dashed] (goal) -- (r1);
    \draw[->, thick, dashed] (r1) -- (c1);
    \draw[->, thick, dashed] (r1) -- (c2);
\end{tikzpicture}
\end{columns}
\end{frame}

\begin{frame}{Chaînage Mixte}
\textbf{Combinaison des deux approches:}
\begin{enumerate}
    \item \textbf{Phase 1}: Chaînage avant pour déduire tous les faits possibles
    \item \textbf{Phase 2}: Chaînage arrière pour vérifier les vulnérabilités restantes
\end{enumerate}

\vspace{0.5cm}
\begin{center}
\begin{tikzpicture}[scale=0.8]
    \node[draw, fill=blue!20, rounded corners] (phase1) at (0,0) {Phase 1: Avant};
    \node[draw, fill=green!20, rounded corners] (phase2) at (5,0) {Phase 2: Arrière};
    \node[draw, fill=orange!20, rounded corners] (result) at (10,0) {Résultats};
    
    \draw[->, thick] (phase1) -- (phase2);
    \draw[->, thick] (phase2) -- (result);
\end{tikzpicture}
\end{center}
\end{frame}

% ===== SECTION 4 =====
\section{Démonstration}

\begin{frame}{Interface - Scanner}
\begin{center}
\textit{[Capture d'écran du Scanner]}

\vspace{0.5cm}
\begin{itemize}
    \item Interface questionnaire en 8 étapes
    \item Sélection des caractéristiques du serveur
    \item Choix de la méthode de chaînage
\end{itemize}
\end{center}
\end{frame}

\begin{frame}{Interface - Résultats}
\begin{center}
\textit{[Capture d'écran des Résultats]}

\vspace{0.5cm}
\begin{itemize}
    \item Vulnérabilités classées par sévérité
    \item Recommandations de correction
    \item Trace d'inférence détaillée
\end{itemize}
\end{center}
\end{frame}

% ===== SECTION 5 =====
\section{Exemple d'Exécution}

\begin{frame}{Exemple: Détection SSH Brute Force}
\textbf{Faits initiaux:}
\begin{itemize}
    \item \texttt{port\_22\_open}
    \item \texttt{password\_auth\_enabled}
\end{itemize}

\vspace{0.3cm}
\textbf{Règle PORT-01:}
\begin{center}
\fbox{\texttt{port\_22\_open} $\land$ \texttt{password\_auth\_enabled} $\Rightarrow$ \texttt{ssh\_brute\_force\_risk}}
\end{center}

\vspace{0.3cm}
\textbf{Résultat:}
\begin{itemize}
    \item[\textcolor{orange}{$\bullet$}] Sévérité: \textbf{DANGEROUS}
    \item Recommandation: Désactiver l'authentification par mot de passe
\end{itemize}
\end{frame}

% ===== SECTION 6 =====
\section{Conclusion}

\begin{frame}{Conclusion}
\textbf{Réalisations:}
\begin{itemize}
    \item Système expert fonctionnel avec 50 règles
    \item 3 méthodes d'inférence implémentées
    \item Interface web moderne et intuitive
    \item Déploiement cloud (Railway + GitHub Pages)
\end{itemize}

\vspace{0.5cm}
\textbf{Améliorations futures:}
\begin{itemize}
    \item Scan automatique des serveurs
    \item Apprentissage de nouvelles règles
    \item Intégration avec des outils de sécurité
\end{itemize}
\end{frame}

\begin{frame}
\begin{center}
\Huge\textbf{Merci pour votre attention}

\vspace{1cm}
\large Questions?

\vspace{1cm}
\normalsize
\texttt{https://kirazul.github.io/Vultester\_polytechnique/}
\end{center}
\end{frame}

\end{document}
