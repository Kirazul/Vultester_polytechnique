\documentclass[12pt,a4paper]{article}
\usepackage[utf8]{inputenc}
\usepackage[T1]{fontenc}
\usepackage[french]{babel}
\usepackage{graphicx}
\usepackage{geometry}
\usepackage{fancyhdr}
\usepackage{titlesec}
\usepackage{enumitem}
\usepackage{booktabs}
\usepackage{longtable}
\usepackage{xcolor}
\usepackage{listings}
\usepackage{hyperref}
\usepackage{float}
\usepackage{tikz}
\usetikzlibrary{shapes.geometric, arrows, positioning}

\geometry{margin=2.5cm}
\hypersetup{colorlinks=true, linkcolor=blue, urlcolor=blue}

% Styles pour le code
\lstset{
    basicstyle=\ttfamily\small,
    breaklines=true,
    frame=single,
    backgroundcolor=\color{gray!10},
    keywordstyle=\color{blue},
    commentstyle=\color{green!60!black},
    stringstyle=\color{red!60!black}
}

% En-tête et pied de page
\pagestyle{fancy}
\fancyhf{}
\fancyhead[L]{Vultester - Système Expert}
\fancyhead[R]{4INFO 2025-2026}
\fancyfoot[C]{\thepage}

\begin{document}

% Page de garde
\begin{titlepage}
    \centering
    \vspace*{1cm}
    
    % Logos
    \begin{figure}[H]
        \centering
        \includegraphics[height=2.5cm]{poly.png}
        \hspace{2cm}
        \includegraphics[height=2.5cm]{logo.png}
    \end{figure}
    
    \vspace{1cm}
    
    {\Large \textbf{École Polytechnique de Sousse}}\\[0.3cm]
    {\large Département Informatique}\\[0.3cm]
    {\large 4ème Année - 2025/2026}
    
    \vspace{1.5cm}
    
    \rule{\textwidth}{1pt}\\[0.5cm]
    {\Huge \textbf{Vultester}}\\[0.3cm]
    {\LARGE Système Expert de Détection de\\Vulnérabilités Serveur}\\[0.5cm]
    \rule{\textwidth}{1pt}
    
    \vspace{1.5cm}
    
    {\large \textbf{Module:} Fondements de l'Intelligence Artificielle}
    
    \vspace{2cm}
    
    \begin{minipage}{0.45\textwidth}
        \begin{flushleft}
            \textbf{Réalisé par:}\\
            Mohamed Aziz Mansour
        \end{flushleft}
    \end{minipage}
    \hfill
    \begin{minipage}{0.45\textwidth}
        \begin{flushright}
            \textbf{Tuteur:}\\
            Mme F. SBIAA
        \end{flushright}
    \end{minipage}
    
    \vfill
    
    {\large Année Universitaire 2025-2026}
\end{titlepage}

% Table des matières
\tableofcontents
\newpage

% ============================================================================
% INTRODUCTION
% ============================================================================
\section{Introduction}

\subsection{Contexte}
La sécurité informatique est devenue un enjeu majeur pour toutes les organisations. Un serveur mal configuré peut devenir une cible facile pour les cyberattaques: ports ouverts inutilement, versions de logiciels obsolètes, absence de chiffrement SSL/TLS, permissions incorrectes sur les fichiers, etc.

Les administrateurs système s'appuient traditionnellement sur leur expertise pour analyser les configurations et identifier les risques. Cependant, cette approche manuelle est:
\begin{itemize}
    \item Chronophage et sujette aux erreurs humaines
    \item Difficile à standardiser et à reproduire
    \item Dépendante de l'expérience individuelle
\end{itemize}

\subsection{Objectifs du Projet}
Ce projet vise à développer un \textbf{système expert} capable de:
\begin{enumerate}
    \item Identifier automatiquement les vulnérabilités d'un serveur
    \item Formaliser l'expertise en sécurité dans une base de connaissances
    \item Implémenter un moteur d'inférence (chaînage avant, arrière, mixte)
    \item Fournir des diagnostics avec 4 niveaux de sévérité
    \item Proposer des actions correctives (patches)
    \item Expliquer le raisonnement via une trace d'inférence
\end{enumerate}

\subsection{Solution Proposée}
\textbf{Vultester} est une application web composée de:
\begin{itemize}
    \item Un \textbf{backend Python/Flask} contenant le moteur d'inférence et 50 règles expertes
    \item Un \textbf{frontend React} offrant une interface moderne et intuitive
    \item \textbf{3 méthodes de chaînage}: avant, arrière et mixte
\end{itemize}

\newpage

% ============================================================================
% CONCEPTION DU SYSTÈME
% ============================================================================
\section{Conception du Système}

\subsection{Architecture Globale}

\begin{figure}[H]
\centering
\begin{tikzpicture}[
    node distance=1.5cm,
    box/.style={rectangle, draw, rounded corners, minimum width=3cm, minimum height=1cm, align=center, fill=blue!10},
    arrow/.style={->, thick}
]
    % Nodes
    \node[box] (user) {Utilisateur};
    \node[box, right=2cm of user] (frontend) {Frontend\\React/TypeScript};
    \node[box, right=2cm of frontend] (api) {API REST\\Flask};
    \node[box, below=1.5cm of api] (engine) {Moteur\\d'Inférence};
    \node[box, left=2cm of engine] (kb) {Base de\\Connaissances\\(50 règles)};
    \node[box, right=2cm of engine] (results) {Résultats\\Diagnostic};
    
    % Arrows
    \draw[arrow] (user) -- (frontend);
    \draw[arrow] (frontend) -- (api);
    \draw[arrow] (api) -- (engine);
    \draw[arrow] (kb) -- (engine);
    \draw[arrow] (engine) -- (results);
    \draw[arrow] (results) -- (api);
\end{tikzpicture}
\caption{Architecture du système Vultester}
\end{figure}

\subsection{Structure des Données}

\subsubsection{Représentation des Règles}
Chaque règle est représentée par un dictionnaire Python:

\begin{lstlisting}[language=Python]
{
    "id": "R1",
    "conditions": ["port_22_open", "password_auth_enabled"],
    "consequence": "ssh_brute_force_risk",
    "severity": "dangerous",
    "description": "SSH avec authentification par mot de passe..."
}
\end{lstlisting}

\subsubsection{Niveaux de Sévérité}
\begin{table}[H]
\centering
\renewcommand{\arraystretch}{1.3}
\begin{tabular}{|l|l|l|}
\hline
\textbf{Niveau} & \textbf{Description} & \textbf{Action} \\
\hline
CRITICAL & Vulnérabilité critique & Action immédiate requise \\
\hline
DANGEROUS & Configuration dangereuse & Action recommandée \\
\hline
WARNING & Avertissement & Révision recommandée \\
\hline
ACCEPTABLE & Configuration acceptable & Surveillance continue \\
\hline
\end{tabular}
\caption{Niveaux de sévérité du diagnostic}
\end{table}

\newpage

% ============================================================================
% MOTEUR D'INFÉRENCE
% ============================================================================
\section{Moteur d'Inférence}

Le système implémente \textbf{trois méthodes de chaînage}, codées manuellement sans bibliothèques spécialisées.

\subsection{Chaînage Avant (Forward Chaining)}

\subsubsection{Principe}
Le chaînage avant part des \textbf{faits connus} (configuration serveur) et applique les règles pour déduire de \textbf{nouveaux faits} (vulnérabilités).

\subsubsection{Algorithme}
\begin{lstlisting}[language=Python]
def forward_chaining(initial_facts):
    queue = copy.deepcopy(initial_facts)
    facts = []
    fired_rules = []
    
    while queue:
        current_fact = queue.pop(0)
        if current_fact not in facts:
            facts.append(current_fact)
        
        for rule in rules:
            if rule not in fired_rules:
                if all(cond in facts for cond in rule.conditions):
                    queue.append(rule.consequence)
                    fired_rules.append(rule)
    
    return analyze_results()
\end{lstlisting}

\subsubsection{Exemple d'Exécution}
\begin{enumerate}
    \item Faits initiaux: \texttt{[port\_22\_open, password\_auth\_enabled]}
    \item Règle R1 vérifiée: conditions satisfaites
    \item Nouveau fait déduit: \texttt{ssh\_brute\_force\_risk}
    \item Diagnostic: Configuration DANGEREUSE
\end{enumerate}

\subsection{Chaînage Arrière (Backward Chaining)}

\subsubsection{Principe}
Le chaînage arrière part des \textbf{buts} (vulnérabilités possibles) et vérifie si les faits peuvent les \textbf{prouver}.

\subsubsection{Algorithme}
\begin{lstlisting}[language=Python]
def backward_chaining(initial_facts, goals):
    for goal in goals:
        if prove_goal(goal, facts, visited=set()):
            proven_goals.append(goal)
    return analyze_results()

def prove_goal(goal, facts, visited):
    if goal in facts:
        return True
    for rule in rules:
        if rule.consequence == goal:
            if all(prove_goal(cond, facts, visited) 
                   for cond in rule.conditions):
                return True
    return False
\end{lstlisting}

\newpage

\subsection{Chaînage Mixte (Mixed Chaining)}

\subsubsection{Principe}
Le chaînage mixte \textbf{combine} les deux approches:
\begin{enumerate}
    \item \textbf{Phase 1}: Chaînage avant pour déduire tous les faits possibles
    \item \textbf{Phase 2}: Chaînage arrière pour vérifier les vulnérabilités non détectées
\end{enumerate}

\subsubsection{Avantages}
\begin{itemize}
    \item Analyse plus \textbf{exhaustive}
    \item Détecte des vulnérabilités que le chaînage avant seul pourrait manquer
    \item Combine l'efficacité des deux méthodes
\end{itemize}

\subsection{Diagramme de Flux du Moteur d'Inférence}

\begin{figure}[H]
\centering
\begin{tikzpicture}[
    node distance=1.2cm,
    startstop/.style={rectangle, rounded corners, draw, fill=red!20, minimum width=2.5cm, minimum height=0.8cm, align=center},
    process/.style={rectangle, draw, fill=blue!20, minimum width=3cm, minimum height=0.8cm, align=center},
    decision/.style={diamond, draw, fill=green!20, minimum width=2cm, minimum height=1cm, align=center, aspect=2},
    arrow/.style={->, thick}
]
    \node[startstop] (start) {Début};
    \node[process, below=of start] (init) {Initialiser faits};
    \node[process, below=of init] (pop) {Extraire fait de la queue};
    \node[decision, below=of pop] (empty) {Queue vide?};
    \node[process, right=2cm of empty] (check) {Vérifier règles};
    \node[decision, below=of check] (satisfied) {Conditions\\satisfaites?};
    \node[process, below=of satisfied] (add) {Ajouter conséquence};
    \node[process, left=2cm of empty] (analyze) {Analyser résultats};
    \node[startstop, below=of analyze] (end) {Fin};
    
    \draw[arrow] (start) -- (init);
    \draw[arrow] (init) -- (pop);
    \draw[arrow] (pop) -- (empty);
    \draw[arrow] (empty) -- node[above] {Non} (check);
    \draw[arrow] (check) -- (satisfied);
    \draw[arrow] (satisfied) -- node[right] {Oui} (add);
    \draw[arrow] (add.west) -- ++(-1,0) |- (pop);
    \draw[arrow] (satisfied.east) -- ++(1,0) node[right] {Non} |- (pop);
    \draw[arrow] (empty) -- node[above] {Oui} (analyze);
    \draw[arrow] (analyze) -- (end);
\end{tikzpicture}
\caption{Diagramme de flux du chaînage avant}
\end{figure}

\newpage

% ============================================================================
% BASE DE CONNAISSANCES
% ============================================================================
\section{Base de Connaissances}

La base contient \textbf{50 règles expertes} organisées en 7 catégories.

\subsection{Catégories de Règles}

\begin{table}[H]
\centering
\renewcommand{\arraystretch}{1.3}
\begin{tabular}{|l|c|l|}
\hline
\textbf{Catégorie} & \textbf{Nb} & \textbf{Description} \\
\hline
Ports critiques & 10 & Ports réseau exposés (SSH, Telnet, FTP...) \\
\hline
SSL/TLS & 8 & Configuration du chiffrement \\
\hline
Configuration SSH & 8 & Accès distant sécurisé \\
\hline
Permissions fichiers & 8 & Droits sur les fichiers système \\
\hline
Versions logiciels & 8 & Services obsolètes \\
\hline
Réseau/Pare-feu & 8 & Configuration réseau \\
\hline
\textbf{Total} & \textbf{50} & \\
\hline
\end{tabular}
\caption{Répartition des règles par catégorie}
\end{table}

\subsection{Exemples de Règles}

\subsubsection{Règles Critiques (CRITICAL)}

\begin{table}[H]
\centering
\renewcommand{\arraystretch}{1.3}
\begin{tabular}{|c|l|l|}
\hline
\textbf{ID} & \textbf{Condition} & \textbf{Description} \\
\hline
PORT-02 & port\_23\_open & Telnet transmet les données en clair \\
\hline
PORT-03 & port\_21\_open ET ftp\_anonymous & FTP anonyme permet accès non autorisé \\
\hline
SSL-06 & ssl\_3\_enabled & SSL 3.0 vulnérable à POODLE \\
\hline
SSH-02 & ssh\_protocol\_1 & SSH v1 a des faiblesses crypto \\
\hline
PERM-01 & permission\_777\_found & Permissions 777 dangereuses \\
\hline
SOFT-01 & apache\_outdated & Apache obsolète avec CVE connus \\
\hline
NET-01 & no\_firewall\_enabled & Pas de pare-feu \\
\hline
\end{tabular}
\caption{Exemples de règles critiques}
\end{table}

\subsubsection{Règles Dangereuses (DANGEROUS)}

\begin{table}[H]
\centering
\renewcommand{\arraystretch}{1.3}
\begin{tabular}{|c|l|l|}
\hline
\textbf{ID} & \textbf{Condition} & \textbf{Description} \\
\hline
PORT-01 & port\_22\_open ET password\_auth & SSH vulnérable au brute force \\
\hline
SSL-01 & no\_ssl\_enabled ET port\_80\_open & HTTP sans HTTPS \\
\hline
SSL-02 & ssl\_certificate\_expired & Certificat SSL expiré \\
\hline
SSH-01 & ssh\_root\_login\_enabled & Login root SSH autorisé \\
\hline
NET-03 & ip\_forwarding\_enabled & IP forwarding actif \\
\hline
SSL-04 & ssl\_weak\_cipher & Chiffrements SSL faibles \\
\hline
SSL-05 & tls\_1\_0\_enabled & TLS 1.0 vulnérable (BEAST, POODLE) \\
\hline
SSH-08 & ssh\_weak\_mac & Algorithmes MAC faibles \\
\hline
NET-04 & syn\_cookies\_disabled & Vulnérable aux SYN floods \\
\hline
NET-08 & snmp\_public\_community & SNMP avec community par défaut \\
\hline
\end{tabular}
\caption{Exemples de règles dangereuses}
\end{table}

\subsubsection{Règles d'Avertissement (WARNING)}

\begin{table}[H]
\centering
\renewcommand{\arraystretch}{1.3}
\begin{tabular}{|c|l|l|}
\hline
\textbf{ID} & \textbf{Condition} & \textbf{Description} \\
\hline
SSL-03 & ssl\_self\_signed & Certificat auto-signé \\
\hline
SSL-07 & no\_hsts\_header & Header HSTS manquant \\
\hline
SSH-04 & ssh\_x11\_forwarding & X11 forwarding peut fuiter des infos \\
\hline
SSH-05 & ssh\_agent\_forwarding & Agent forwarding exploitable \\
\hline
SSH-07 & ssh\_no\_fail2ban & Pas de protection brute force \\
\hline
PERM-08 & tmp\_not\_secured & /tmp non sécurisé \\
\hline
NET-02 & icmp\_enabled ET no\_rate\_limiting & Ping flood possible \\
\hline
\end{tabular}
\caption{Exemples de règles d'avertissement}
\end{table}

\subsubsection{Règles Informatives (INFO)}

\begin{table}[H]
\centering
\renewcommand{\arraystretch}{1.3}
\begin{tabular}{|c|l|l|}
\hline
\textbf{ID} & \textbf{Condition} & \textbf{Description} \\
\hline
SSH-06 & ssh\_default\_port & SSH sur port 22 par défaut \\
\hline
\end{tabular}
\caption{Règles informatives}
\end{table}

\newpage

\subsection{Justification des Règles}

Les règles sont basées sur:
\begin{itemize}
    \item \textbf{CVE (Common Vulnerabilities and Exposures)}: Base de données des vulnérabilités connues
    \item \textbf{CIS Benchmarks}: Standards de sécurité de l'industrie
    \item \textbf{OWASP}: Recommandations de sécurité web
    \item \textbf{NIST}: Guidelines de cybersécurité
\end{itemize}

\subsubsection{Exemple: Règle PORT-01 - SSH Brute Force}
\begin{itemize}
    \item \textbf{Condition}: Port 22 ouvert + Authentification par mot de passe
    \item \textbf{Risque}: Attaques par force brute sur les mots de passe
    \item \textbf{Source}: CIS Benchmark for Linux - Section 5.2.8
    \item \textbf{Recommandation}: Utiliser l'authentification par clé SSH
\end{itemize}

\subsubsection{Exemple: Règle PERM-01 - Permissions 777}
\begin{itemize}
    \item \textbf{Condition}: Fichiers avec permissions 777
    \item \textbf{Risque}: Tout utilisateur peut lire, écrire et exécuter
    \item \textbf{Source}: CIS Benchmark - Section 6.1
    \item \textbf{Recommandation}: Restreindre les permissions (chmod 755 ou moins)
\end{itemize}

\subsection{Recommandations (Patches)}

Chaque vulnérabilité détectée est accompagnée d'une \textbf{action corrective}:

\begin{table}[H]
\centering
\renewcommand{\arraystretch}{1.3}
\begin{tabular}{|l|l|}
\hline
\textbf{Vulnérabilité} & \textbf{Recommandation} \\
\hline
ssh\_brute\_force\_risk & PasswordAuthentication no dans sshd\_config \\
\hline
telnet\_vulnerability & systemctl disable telnet \\
\hline
root\_ssh\_risk & PermitRootLogin no \\
\hline
world\_writable\_files & chmod 755 sur les fichiers concernés \\
\hline
no\_firewall\_protection & ufw enable \\
\hline
\end{tabular}
\caption{Exemples de recommandations}
\end{table}

\newpage

% ============================================================================
% INTERFACE UTILISATEUR
% ============================================================================
\section{Interface Utilisateur}

\subsection{Technologies Utilisées}

\begin{itemize}
    \item \textbf{React 18}: Framework JavaScript moderne
    \item \textbf{TypeScript}: Typage statique pour la fiabilité
    \item \textbf{Tailwind CSS}: Framework CSS utilitaire
    \item \textbf{Framer Motion}: Animations fluides
    \item \textbf{React Router}: Navigation entre pages
    \item \textbf{Vite}: Build tool rapide
\end{itemize}

\subsection{Pages de l'Application}

\subsubsection{Page d'Accueil (Home)}
\begin{itemize}
    \item Présentation du système
    \item Statistiques: 50 règles, 4 niveaux, 7 catégories
    \item Boutons d'action: Lancer l'analyse, Voir les règles
\end{itemize}

\subsubsection{Scanner de Vulnérabilités}
Interface de type \textbf{survey} en 8 étapes:
\begin{enumerate}
    \item Ports ouverts (10 options)
    \item Configuration SSL/TLS (9 options)
    \item Configuration SSH (9 options)
    \item Permissions fichiers (6 options)
    \item Versions logiciels (6 options)
    \item Base de données (8 options)
    \item Configuration réseau (8 options)
    \item \textbf{Choix de la méthode de chaînage}
\end{enumerate}

\subsubsection{Base de Connaissances}
\begin{itemize}
    \item Liste des 50 règles avec filtrage
    \item Recherche par mot-clé
    \item Filtrage par sévérité
    \item Affichage format: SI conditions ALORS conséquence
\end{itemize}

\subsubsection{Page À Propos}
\begin{itemize}
    \item Description du système
    \item Architecture technique
    \item Crédits (étudiant, tuteur, école)
\end{itemize}

\newpage

\subsection{Captures d'Écran}

\begin{figure}[H]
\centering
\includegraphics[width=0.9\textwidth]{screenshot-home.png}
\caption{Page d'accueil de Vultester}
\end{figure}

\begin{figure}[H]
\centering
\includegraphics[width=0.9\textwidth]{screenshot-scanner.png}
\caption{Interface du scanner de vulnérabilités}
\end{figure}

\begin{figure}[H]
\centering
\includegraphics[width=0.9\textwidth]{screenshot-results.png}
\caption{Affichage des résultats avec diagnostic}
\end{figure}

\begin{figure}[H]
\centering
\includegraphics[width=0.9\textwidth]{screenshot-rules.png}
\caption{Liste des règles expertes}
\end{figure}

\newpage

% ============================================================================
% EXEMPLE D'EXÉCUTION
% ============================================================================
\section{Exemple d'Exécution}

\subsection{Scénario de Test}

\textbf{Configuration serveur simulée:}
\begin{itemize}
    \item Port 22 (SSH) ouvert
    \item Port 80 (HTTP) ouvert
    \item Authentification SSH par mot de passe
    \item SSL non activé
    \item Permissions 777 détectées
    \item Pas de pare-feu
\end{itemize}

\subsection{Faits Initiaux}
\begin{lstlisting}
facts = [
    "port_22_open",
    "port_80_open", 
    "password_auth_enabled",
    "no_ssl_enabled",
    "permission_777_found",
    "no_firewall_enabled"
]
\end{lstlisting}

\subsection{Trace d'Inférence (Chaînage Avant)}

\begin{enumerate}
    \item \textbf{Étape 0}: Initialisation avec 6 faits
    \item \textbf{Étape 1}: Fait ajouté: port\_22\_open
    \item \textbf{Étape 2}: Fait ajouté: password\_auth\_enabled
    \item \textbf{Étape 3}: Règle PORT-01 déclenchée $\rightarrow$ ssh\_brute\_force\_risk
    \item \textbf{Étape 4}: Fait ajouté: no\_ssl\_enabled
    \item \textbf{Étape 5}: Règle SSL-01 déclenchée $\rightarrow$ unencrypted\_traffic
    \item \textbf{Étape 6}: Règle PERM-01 déclenchée $\rightarrow$ world\_writable\_files
    \item \textbf{Étape 7}: Règle NET-01 déclenchée $\rightarrow$ no\_firewall\_protection
\end{enumerate}

\subsection{Résultat du Diagnostic}

\begin{table}[H]
\centering
\renewcommand{\arraystretch}{1.3}
\begin{tabular}{|l|c|}
\hline
\textbf{Métrique} & \textbf{Valeur} \\
\hline
Statut global & \textcolor{red}{\textbf{CRITICAL}} \\
\hline
Règles déclenchées & 4 \\
\hline
Vulnérabilités critiques & 2 \\
\hline
Configurations dangereuses & 2 \\
\hline
Avertissements & 0 \\
\hline
\end{tabular}
\caption{Résumé du diagnostic}
\end{table}

\newpage

% ============================================================================
% LIMITES ET AMÉLIORATIONS
% ============================================================================
\section{Limites et Améliorations Possibles}

\subsection{Limites Actuelles}

\begin{enumerate}
    \item \textbf{Détection manuelle}: L'utilisateur doit saisir manuellement la configuration
    \item \textbf{Règles statiques}: Les règles ne s'adaptent pas automatiquement aux nouvelles vulnérabilités
    \item \textbf{Pas de scan réseau}: Le système ne scanne pas directement les serveurs
    \item \textbf{Base de connaissances limitée}: 50 règles couvrent les cas courants mais pas exhaustifs
    \item \textbf{Pas de persistance}: Les analyses ne sont pas sauvegardées
\end{enumerate}

\subsection{Améliorations Possibles}

\begin{enumerate}
    \item \textbf{Scan automatique}: Intégrer Nmap pour détecter automatiquement les ports ouverts
    \item \textbf{Base de données CVE}: Connecter à une API CVE pour des règles à jour
    \item \textbf{Machine Learning}: Utiliser l'apprentissage pour améliorer les diagnostics
    \item \textbf{Historique}: Sauvegarder les analyses pour suivre l'évolution
    \item \textbf{Rapports PDF}: Générer des rapports exportables
    \item \textbf{Multi-serveurs}: Analyser plusieurs serveurs simultanément
    \item \textbf{Alertes}: Notifications en cas de nouvelles vulnérabilités
\end{enumerate}

\newpage

% ============================================================================
% CONCLUSION
% ============================================================================
\section{Conclusion}

Ce projet a permis de développer \textbf{Vultester}, un système expert complet pour la détection de vulnérabilités serveur. Les objectifs ont été atteints:

\begin{itemize}
    \item[$\checkmark$] \textbf{Base de connaissances}: 50 règles expertes (requis: 15 minimum)
    \item[$\checkmark$] \textbf{Moteur d'inférence}: 3 méthodes implémentées (avant, arrière, mixte)
    \item[$\checkmark$] \textbf{4 niveaux de diagnostic}: Critical, Dangerous, Warning, Acceptable
    \item[$\checkmark$] \textbf{Recommandations}: Actions correctives pour chaque vulnérabilité
    \item[$\checkmark$] \textbf{Trace d'inférence}: Explication du raisonnement étape par étape
    \item[$\checkmark$] \textbf{Interface web}: Application React moderne et intuitive
    \item[$\checkmark$] \textbf{Code manuel}: Moteur d'inférence sans bibliothèques spécialisées
\end{itemize}

Le système démontre l'application pratique des concepts de systèmes experts vus en cours, notamment le chaînage avant inspiré de l'algorithme \texttt{chainageAvantSimple}.

\vspace{1cm}

\begin{center}
\rule{0.5\textwidth}{0.5pt}
\end{center}

% ============================================================================
% RÉFÉRENCES
% ============================================================================
\section*{Références}

\begin{enumerate}
    \item CIS Benchmarks - Center for Internet Security
    \item OWASP - Open Web Application Security Project
    \item NIST Cybersecurity Framework
    \item CVE - Common Vulnerabilities and Exposures Database
    \item Cours "Fondements de l'IA" - 4INFO 2025-2026
\end{enumerate}

\end{document}
